\documentclass[11pt]{article}

\usepackage[T1]{fontenc}
\usepackage{geometry}
\usepackage{amsmath, amssymb, amsthm}
\usepackage{xcolor}
\usepackage{graphicx}

\geometry{a4paper, bottom = 1in}

\newcommand{\iv}{$^{-1}$}

\title{A model for in-host viral infection dynamics}
\author{Satvik Saha}
\date{}

\begin{document}
    \maketitle

    \section{Model descriptions}

    \subsection{Individual based model}

    This model accommodates $n$ individuals, each of which belongs to one of
    $m$ environments. The individuals also form the nodes of a contact
    network. Each individual maintains its own in-host variables $T, T^*, V,
    A$ which evolve over time. Each environment accumulates viral copies from
    its members, which decay over time. At the start of every day, each
    individual is stochastically assigned a subset of its contact neighbours,
    which together with the individual's environment contribute an `external
    pressure' of viral copies. If this crosses a threshold $v$, then the
    individual becomes infected and experiences a rapid increase in its
    in-host viral load $V$.

    ~

    Let $n, m \in \mathbb{N}$. For each $i \in \{1, 2, \dots, n\}$ and $j \in
    \{1, 2, \dots, m\}$, let

    \begin{align}
        \frac{dT_i}{dt} &= b -  \delta T_i - \frac{\kappa}{1 + \alpha A_i} T_iV_i, \\
        \frac{dT^*_i}{dt} &= \frac{\kappa}{1 + \alpha A_i} T_iV_i - qT^*_i, \\
        \frac{dV_i}{dt} &= pT^*_i - cV_i - c_A A_iV_i - X_i(t) + g\left(W_i + \sum_{j = 1}^m \eta_{ij} Z_j\right), \\
        \frac{dA_i}{dt} &= b_A -  \delta_A A_i + \kappa_A A_i(t - \tau)V_i(t - \tau), \\\\
        \frac{dZ_j}{dt} &= \sum_{i = 1}^n \xi_{ij} V_i - \delta_Z Z_j, \\
        W_i(t) &= \zeta \sum_{k = 1}^n Y_{ik}(\lfloor t\rfloor) V_k(\lfloor t\rfloor), \\
        g(x) &= \begin{cases}
            x, &\text{ if } x > v, \\
            0, &\text{ if } x \leq v.
        \end{cases}.
    \end{align}

    Here, we define random variables $X_i(t) \sim
    \operatorname{Exp}(\lambda)$ drawn independently for each $t \geq 0$, and
    $Y_{ik}(\ell) \sim \operatorname{Bernoulli}(s_{ik} \, p_\text{inf})$ drawn
    independently for each $\ell \in \mathbb{Z}_{\geq 0}$.

    ~

    The parameters $\xi_{ij}, \eta_{ij}$ are to be thought of as weights
    linking individuals with their environments; the parameters $s_{ik}$ are
    to be thought of as strengths of connections between individuals forming a
    network.

    ~

    The role of the stochastic term $X_i(t)$ in the equation for $dV_i / dt$
    is to allow the possibility of \textit{complete} removal of viral load
    $V_i$ from an individual. This stochastic effect is prominent when $V_i
    \sim 1/\lambda$, and becomes negligible when $V_i \gg 1 / \lambda$.  This
    is introduced to prevent $V_i$ from `bouncing back' after an initial
    infection.

    ~

    The model state is described by $(T, T^*, V, A, W, Z) \in \mathcal{S}
    \equiv \mathbb{R}_{\geq 0}^{3n} \times \mathcal{C}_\tau \times
    \mathcal{C}_\tau \times \mathbb{R}_{\geq 0}^{m}$, with $\mathcal{C}_\tau
    \equiv \mathcal{C}(0, \tau)$, $T \equiv (T_1, \dots, T_n)$, and so on.

    The model parameters are $(b, \delta, \kappa, q, p, c, b_A, \delta_A,
    \kappa_A, c_A, \alpha, \tau, \delta_Z, v, \lambda, p_\text{inf}, \zeta,
    \eta, \xi, s) \in \mathcal{P} \equiv \mathbb{R}_{\geq 0}^{16} \times
    [0, 1] \times \mathbb{R}_{\geq 0}^{mn} \times \mathbb{R}_{\geq 0}^{mn}
    \times [0, 1]^{n\times n}$, with $\eta \equiv [\eta_{ij}]_{ij}$, $\xi
    \equiv [\xi_{ij}]_{ij}$, and $s \equiv [s_{ik}]_{ik}$.

    ~

    \paragraph{Simplifications:} Given $i$, let $\eta_{ij} = \eta$, $\xi_{ij}
    = \xi$ for precisely one $j$ and $\eta_{ij}, \xi_{ij} = 0$ for the rest.
    In other words, let each individual belong to precisely one environment.
    Furthermore, let $s_{ik} \in \{0, 1\}$, with all $s_{ii} = 0$.

    With the underlying individual-environment and individual-individual
    connections fixed, the parameter space reduces to $\mathbb{R}_{\geq
    0}^{19}$.



    \begin{table}[h!]
        \centering
        \caption{Model state variables}
        \vspace{1em}
        \begin{tabular}{ccl} \hline\hline
            Variable    & Units     & Interpretation \\\hline\hline
            $T$         & cells/ml  & Concentration of target cells \\
            $T^*$       & cells/ml  & Concentration of infected cells \\
            $V$         & copies/ml & Concentration of viral copies \\
            $A$         & imm/ml    & Antibody/immunity level \\
            $W$         & copies/ml
                                    & Contact pressure of viral copies \\
            $Z$         & copies/m$^2$
                                    & Environmental viral copies \\
            \hline
        \end{tabular}
        \label{tab:state}
    \end{table}

    \begin{table}[h!]
        \centering
        \caption{Model parameters}
        \vspace{1em}
        \begin{tabular}{ccl} \hline\hline
            Parameter   & Units     & Interpretation \\\hline\hline
            $b$         & cells ml\iv day\iv
                                    & Generation rate of target cells \\
            $\delta$    & day\iv
                                    & Death rate of target cells \\
            $\kappa$    & cells\iv ml day\iv
                                    & Infection rate of target cells \\
            $q$         & day\iv
                                    & Death rate of infected cells \\
            $p$         & copies cells\iv day\iv
                                    & Production rate of viral copies \\
            $c$         & day\iv
                                    & Clearance rate of viral copies \\
            $b_A$       & imm ml\iv day\iv
                                    & Generation rate of antibodies \\
            $\delta_A$  & day\iv
                                    & Clearance rate of antibodies \\
            $\kappa_A$  & copies\iv ml day\iv
                                    & Production rate of antibodies \\
            $c_A$       & imm\iv ml day\iv
                                    & Clearance rate of viral copies via antibodies \\
            $\alpha$    & imm\iv ml
                                    & Inhibition of viral-target contact \\
            $\tau$      & day
                                    & Delay in antibody production \\
            $\delta_Z$  & day\iv
                                    & Removal rate of viral copies \\
            $v$         & copies ml\iv
                                    & Entry threshold of viral concentration \\
            $\lambda$   & copies\iv ml day
                                    & Reciprocal of mean of stochastic viral removal \\
            $p_\text{inf}$  & --
                                    & Probability of viral load transfer \\
            $\zeta$     & day\iv
                                    & Rate of viral load transfer \\
            $\eta_{ij}$ & ml\iv m$^2$ day\iv
                                    & Environment-Individual transmission rate of virus\\
            $\xi_{ij}$  & ml m$^{-2}$ day\iv
                                    & Viral shedding rate into environment \\
            $s_{ik}$    & --
                                    & Strength of contact between individuals \\
            \hline
        \end{tabular}
        \label{tab:parameter}
    \end{table}

    ~

    After choosing thresholds $V'$ and $A'$, we can count \begin{align}
        S &= \sum_{i = 1}^n \mathbf{1}(A \leq A')\, \mathbf{1}(V \leq V') \\
        I &= \sum_{i = 1}^n \mathbf{1}(V > V'), \\
        R &= n - S - I.
    \end{align}

    Here, $S$ denotes the number of susceptible individuals, $I$ denotes the
    number of infectuous individuals, and $R$ denotes the number of recovered
    individuals. The threshold $V'$ is chosen such that its contribution to
    the external pressure of viral copies is enough to cross the barrier $v$.
    The threshold $A'$ is chosen such that a typical individual with that
    level of antibodies is immune to infection.


    \subsection{In-host submodel}

    Consider the in-host model described below.

    \begin{align*}
        \frac{dT}{dt} &= b -  \delta T - \frac{\kappa}{1 + \alpha A} TV, \\
        \frac{dT^*}{dt} &= \frac{\kappa}{1 + \alpha A} TV - qT^*, \\
        \frac{dV}{dt} &= pT^* - cV - c_A AV, \\
        \frac{dA}{dt} &= b_A -  \delta_A A + \kappa_A A(t - \tau)V(t - \tau).
    \end{align*}

    \paragraph{Equilibria:} Note that $(b / \delta, 0, 0, b_A / \delta_A)$ is
    a trivial infection-free equilibrium. Solving for other equilibria, we
    demand \[
        \frac{\kappa}{1 + \alpha A} TV = b - \delta T = q T^*, \qquad
        pT^* = (c + c_A A)V, \qquad
        \delta_A A  - b_A = \kappa_A AV.
    \] Thus, \[
        \frac{p}{q}(b - \delta T) = \frac{(c + c_A A)(\delta_A A - b_A)}{\kappa_A A},
    \] whence \[
        T = \frac{b}{\delta} - \frac{q(c + c_A A)(\delta_A A - b_A)}{p \delta \kappa_A A}.
    \] Furthermore, \[
        1 + \alpha A = \frac{\kappa TV}{b - \delta T} = T \frac{(\delta_A A - b_A) / \kappa_A}{(b - \delta T) / \kappa},
    \] whence \[
        \frac{b - \delta T}{\kappa T} = \frac{\delta_A A - b_A}{\kappa_A (1 + \alpha A)}.
    \] Thus, \[
        T  = \frac{q(c + c_A A)(1 + \alpha A)}{p\kappa A}.
    \] This gives \[
        \frac{b}{\delta} = \frac{q(c + c_A A)}{pA}\left[\frac{1 + \alpha A}{\kappa} + \frac{\delta_A A - b_A}{\delta \kappa_A}\right].
    \] Putting $T_0 = b / \delta$, $A_0 = b_A / \delta_A$, we have \[
        pAT_0 = q(c + c_A A)\left[\frac{1 + \alpha A}{\kappa} + \frac{\delta_A(A - A_0)}{\delta \kappa_A}\right],
    \] whence \[
        p\kappa T_0 A
        = q(c + c_A A)\left[1 - \frac{\kappa/\delta}{\kappa_A/\delta_A}A_0 + \left(\alpha + \frac{\kappa / \delta}{\kappa_A / \delta_A}\right)A\right].
    \] Setting $\beta = (\kappa / \delta) / (\kappa_A / \delta_A)$, $r = p /
    q$, $\gamma = c_A / c$, we have \[
        \kappa r T_0 A = c(1 + \gamma A)[1 - \beta A_0 + (\alpha + \beta)A].
    \] Thus, \[
        \gamma(\alpha + \beta)A^2 + [\gamma(1 - \beta A_0) + (\alpha + \beta) - \kappa r T_0 / c] A + (1 - \beta A_0) = 0,
    \] or \[
        A^2 + \left[\frac{1 - \beta A_0}{\alpha + \beta} + \frac{1}{\gamma} - \frac{\kappa r T_0}{c\gamma(\alpha + \beta)}\right] A + \frac{1 - \beta A_0}{\gamma(\alpha + \beta)} = 0.
    \]


    \subsection{Multiscale model}

    \begin{align}
        \frac{dS}{dt} &= -\beta_I(V, I) SI - \beta_Z(Z) SZ + \mu R, \\
        \frac{dI}{dt} &= \beta_I(V, I) SI + \beta_Z(Z) SZ - \gamma I, \\
        \frac{dR}{dt} &= \gamma I - \mu R, \\\\
        \frac{dZ}{dt} &= \xi I - \delta_Z Z, \\\\
        \epsilon \frac{dT}{dt} &= b -  \delta T - \frac{\kappa}{1 + \alpha A} TV, \\
        \epsilon \frac{dT^*}{dt} &= \frac{\kappa}{1 + \alpha A} TV - qT^*, \\
        \epsilon \frac{dV}{dt} &= \eta Z + pT^* - cV - c_A AV, \\
        \epsilon \frac{dA}{dt} &= b_A -  \delta_A A + \kappa_A A(t - \tau)V(t - \tau).
    \end{align}

    Here, \[
        \beta_I(V, I) = \frac{\beta_{I0} + C_0 V}{1 + C_1 I}, \qquad
        \beta_Z(Z) = \frac{\beta_{Z0}}{1 + C_2 Z}.
    \]


    \subsection{SIRS model}

    \begin{align}
        \frac{dS}{dt} &= -\frac{\beta SI}{N} + \mu R, \\
        \frac{dI}{dt} &= \frac{\beta SI}{N} - \gamma I, \\
        \frac{dR}{dt} &= \gamma I - \mu R.
    \end{align}

    Here, $N = S + I + R$.


    \section{Objectives}

    \begin{enumerate}
        \item Compare the $S, I, R$ curves with those obtained from a
        simplified model with one individual and one environment.

        \item Identify/interpret infection phases ($S$, $I$, $R$) using the
        in-host variables ($T$, $T^*$, $V$, $A$).

        \item Investigate the effects of heterogeneity in the individuals and
        their contact network. For instance,
        \begin{enumerate}
            \item In-host parameters may be varied across individuals, forming
            two or more groups.
            \item Groups of individuals may be vaccinated.
        \end{enumerate}

        \item Investigate the effect of the stochastic term $X_i(t)$ in the
        in-host model.
    \end{enumerate}


    \section{Observations}

    \begin{enumerate}
        \item The individual based model (1.1) is capable of producing
        infection curves with multiple waves/peaks.

        \item Averaged infection curves from model (1.1) also show multiple
        peaks; the curve up to the first peak fits well against the SIRS model
        (1.4). Individual infection curves from model (1.1) up to the first
        peak also fit well against model (1.4).

        \item There is a narrow range of $p_\text{inf}$, with all other
        parameters in model (1.1) fixed, in which a significant proportion of
        infection curves display multiple prominent peaks without damping. For
        lower $p_\text{inf}$, infection curves become more stochastic in
        nature.

        \item Individuals in model (1.1) become `infected' when a pulse is
        applied on $W_i$. The viral load $V_i$ rapidly increases, which after
        a short delay leads to a rapid increase in the antibody/immunity
        $A_i$. This forces $V_i$ to fall sharply to zero, after which $A_i$
        gradually drops back to its baseline level. A sufficiently elevated
        $A_i$ confers `immunity' to the individual, preventing reinfection.
        The probability of reinfection, as a function of time since infection,
        can be calculated.
    \end{enumerate}


    \begin{figure}[h!]
    \begin{center}
        \includegraphics[page = 2, width = \textwidth]{../figures/multirun.pdf}
    \end{center}
    \caption{Infection curves, by varying infection probabilities
    $p_\text{inf}$.}
    \label{fig:multiruns}
    \end{figure}


    \begin{figure}[h!]
    \begin{center}
        \includegraphics[page = 4, width = \textwidth]{../figures/multirun.pdf}
    \end{center}
    \caption{Number of runs (out of 200) with $n$ peaks in the infection
    curve, by varying infection probabilities $p_\text{inf}$.}
    \label{fig:peaks}
    \end{figure}

    \begin{figure}[h!]
    \begin{center}
        \includegraphics[page = 3, width = \textwidth]{../figures/multirun.pdf}
    \end{center}
    \caption{Distribution of epidemic durations, by varying infection
    probabilities $p_\text{inf}$.}
    \label{fig:durations}
    \end{figure}

\end{document}
